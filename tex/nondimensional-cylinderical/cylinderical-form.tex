\section{Derivation of the Non-Dimensional Cylinderical Navier Stokes Equations}

In this section the system of governing equations given previously will be
expanded in a cylinderical coordiniate system and reduced to a more convenient
non-dimensional form.

For cylinderical ($x$,$r$,$\phi$) coordinates, the velocity vector
$\overrightarrow{V}$ is given by:

\begin{equation}
\overrightarrow{V} = u \hat{e_x} + v \hat{e_r} + w \hat{e_\phi}
\end{equation}

where $\hat{e_x}$, $\hat{e_r}$, and $\hat{e_\phi}$ are the unit vectors.

The del $\overrightarrow{\nabla}$ operator may be given as:

\begin{equation}
  \overrightarrow{\nabla} =  
      \hat{e_x}\frac{\partial}{\partial x} 
    + \hat{e_r}\frac{\partial}{\partial r} 
    + \hat{e_\phi}\frac{\partial}{\partial\phi}
\end{equation}

Since the analysis that follows will require taking derivatives of the unit
vectors, these expressions will be derived next.

The position vector $\overrightarrow{r}$ may be written in cartesian coordinates as:

\begin{equation}
  \begin{aligned}
    \overrightarrow{r} &= x\hat{i} + y\hat{j} + z\hat{k} \\
        &= x\hat{i} + r\cos\phi\hat{j} + r\sin\phi\hat{k}
  \end{aligned}
\end{equation}

Now the equations for the unit vectors are:

\begin{equation}
  \hat{e_x} = 
  \frac{\partial\overrightarrow{r}}{\partial x} = \hat{i}
\end{equation}

\begin{equation}
  \hat{e_r} 
  = \frac{\partial\overrightarrow{r}}{\partial{r}} 
  = \cos{\phi}\hat{j} + \sin{\phi}\hat{k}
\end{equation}

\begin{equation}
  \hat{e_\phi} 
  = \frac{\partial\overrightarrow{r}}{\partial{\phi}} 
  = - \sin{\phi}\hat{j} + \cos{\phi}\hat{k}
\end{equation}

From these equations, the derivatives of the unit vectors can be found:

\begin{equation}
    \frac{\partial\hat{e_x}}{\partial x} 
  = \frac{\partial\hat{e_x}}{\partial r} 
  = \frac{\partial\hat{e_x}}{\partial\phi} = 0
\end{equation}

\begin{equation}
  \frac{\partial\hat{e_r}}{\partial x} = \frac{\partial\hat{e_r}}{\partial r} = 0
\end{equation}

\begin{equation}
  \frac{\partial\hat{e_r}}{\partial\phi} = 
  - \sin\phi\hat{j} 
  + \cos\phi\hat{k} = \hat{e_\phi}
\end{equation}

\begin{equation}
  \frac{\partial\hat{e_\phi}}{\partial x} = 
  \frac{\partial\hat{e_\phi}}{\partial r} = 0
\end{equation}

\begin{equation}
  \frac{\partial\hat{e_\phi}}{\partial\phi} = 
  - \cos\phi\hat{j} 
  - \sin\phi\hat{k} = 
  - \hat{e_r}
\end{equation}

Using these expressions, the derivatives of a general vector function become:

\begin{equation}
  \begin{aligned}
    \frac{\overrightarrow{F}}{\partial x} &= 
    \frac{\partial F_1}{\partial x}\hat{e_x} + 
    \frac{\partial F_2}{\partial x}\hat{e_r} + 
    \frac{\partial F_3}{\partial x}\hat{e_\phi} \\
    \frac{\overrightarrow{F}}{\partial r} &= 
    \frac{\partial F_1}{\partial r}\hat{e_x} + 
    \frac{\partial F_2}{\partial r}\hat{e_r} + 
    \frac{\partial F_3}{\partial r}\hat{e_\phi} \\
    \frac{\overrightarrow{F}}{\partial \phi} &= 
    \frac{\partial F_1}{\partial \phi}\hat{e_x} + 
    \left(\frac{\partial F_2}{\partial \phi} - F_3\right)\hat{e_r} + 
    \left(\frac{\partial F_3}{\partial \phi} + F2\right)\hat{e_\phi}
  \end{aligned}
\end{equation}

where $F_1$, $F_2$, and $F_3$ are the vector components.

\subsection{Continuity Equation}

The continuity equation will be expanded first:

\begin{equation}
\frac{D\rho}{Dt} 
  + \rho \overrightarrow{\nabla}\cdot\overrightarrow{V} = 0
\end{equation}

\begin{equation}
  \frac{\partial\rho}{\partial t} 
  + \left(\overrightarrow{V}\cdot\overrightarrow{\nabla}\rho\right)
  + \rho\left(\overrightarrow{\nabla}\cdot\overrightarrow{V}\right) = 0
\end{equation}

\begin{equation}
  \frac{\partial\rho}{\partial t}  
  + u\frac{\partial\rho}{\partial x}
  + v\frac{\partial\rho}{\partial r} 
  + \frac{r}{r}\frac{\partial\rho}{\partial\phi}
  + \rho\left(\frac{\partial u}{\partial x}
  + \frac{\partial v}{\partial r} 
  + \frac{1}{r}\frac{\partial w}{\partial\phi} 
  + \frac{v}{r} \right) = 0
\end{equation}

After rearranging and combining terms, we get this form:

\begin{equation}
  \frac{\partial\rho}{\partial t}
  + \frac{\partial}{\partial x}\left(\rho u\right)
  + \frac{1}{r}\frac{\partial}{\partial r}\left(\rho v r \right)
  + \frac{1}{r}\frac{\partial}{\partial\phi}\left(\rho w\right) = 0
\end{equation}

\subsection{Momentum Equation}

Before expanding the momentum equation, it will be convenient to use the following for the shear stress matrix:

\begin{equation}
  \rttensor{\tau} 
  = \overrightarrow{\tau_1}\hat{e_x} 
  + \overrightarrow{\tau_2}\hat{e_r}
  + \overrightarrow{\tau_3}\hat{e_\phi}
\end{equation}

Where:

\begin{equation}
  \begin{aligned}
    \overrightarrow{\tau_1} 
    &= \tau_{xx}\hat{e_x}
    + \tau_{xr}\hat{e_r}
    + \tau_{x\phi}\hat{e_\phi} \\
     \overrightarrow{\tau_2} 
    &= \tau_{rx}\hat{e_x}
    + \tau_{rr}\hat{e_r}
    + \tau_{r\phi}\hat{e_\phi} \\
     \overrightarrow{\tau_3} 
    &= \tau_{\phi x}\hat{e_x}
    + \tau_{\phi r}\hat{e_r}
    + \tau_{\phi\phi}\hat{e_\phi}
  \end{aligned}
\end{equation}

In this investigation, body forces will be neglected. The momentum equation becomes:

\begin{equation}
  \begin{aligned}
    \rho\frac{D\overrightarrow{V}}{Dt}
    &= \rho\overrightarrow{f_g} 
    + \overrightarrow{\nabla}\cdot\rttensor{\tau} \\
    \rho\{\frac{\overrightarrow{V}}{\partial t} 
    + \overrightarrow{V}\cdot\overrightarrow{\nabla}\overrightarrow{V}\} 
    &= \overrightarrow{\nabla}\cdot\rttensor{\tau}
  \end{aligned}
\end{equation}

or

\begin{equation}
  \begin{aligned}
  \rho \bigg\{ 
      \frac{\partial u}{\partial t}\hat{e_x} 
    + \frac{\partial v}{\partial t}\hat{e_r}
    + \frac{\partial w}{\partial t}\hat{e_\phi} 
    + u\frac{\overrightarrow{V}}{\partial t}\hat{e_x}
    + v\frac{\overrightarrow{V}}{\partial r}\hat{e_r}
    + \frac{w}{r}\frac{\overrightarrow{V}}{\partial \phi}\hat{e_\phi}
  \bigg\} \\
  &= \hat{e_x}\cdot\frac{\partial}{\partial x}
  \{
      \overrightarrow{\tau_1}\hat{e_x}
    + \overrightarrow{\tau_2}\hat{e_r}
    + \overrightarrow{\tau_3}\hat{e_\phi}
  \} \\
  &+ \hat{e_r}\cdot\frac{\partial}{\partial r}
  \{
      \overrightarrow{\tau_1}\hat{e_x}
    + \overrightarrow{\tau_2}\hat{e_r}
    + \overrightarrow{\tau_3}\hat{e_\phi}
  \} \\
  &+ \frac{\hat{e_r}}{r}\cdot\frac{\partial}{\partial \phi}
  \{
      \overrightarrow{\tau_1}\hat{e_x}
    + \overrightarrow{\tau_2}\hat{e_r}
    + \overrightarrow{\tau_3}\hat{e_\phi}
  \} \\
  &= \frac{\partial\overrightarrow{\tau_1}}{\partial x} 
    + \frac{\partial\overrightarrow{\tau_2}}{\partial r} 
    + \frac{1}{r} 
    \bigg\{
      \frac{\partial\overrightarrow{\tau_3}}{\partial \phi}  
      + \overrightarrow{\tau_2}
    \bigg\}
  \end{aligned}
\end{equation}

By expanding this last expression and rearranging terms, the following form
results for the momentum equation

\begin{equation}
  \begin{aligned}
\rho \bigg\{ 
      \frac{\partial u}{\partial t}\hat{e_x} 
    + \frac{\partial v}{\partial t}\hat{e_r}
    + \frac{\partial w}{\partial t}\hat{e_\phi} 
    + u\frac{\overrightarrow{V}}{\partial t}\hat{e_x}
    + v\frac{\overrightarrow{V}}{\partial r}\hat{e_r}
    + \frac{w}{r}\frac{\overrightarrow{V}}{\partial \phi}\hat{e_\phi}
  \bigg\}a \\
    &+ \rho u \bigg\{
    \frac{\partial u}{\partial x}\hat{e_x} 
    + \frac{\partial v}{\partial x}\hat{e_r}
    + \frac{\partial w}{\partial x}\hat{e_\phi} \bigg\}\\
  &+ \rho v \bigg\{
    \frac{\partial u}{\partial r}\hat{e_x} 
    + \frac{\partial v}{\partial r}\hat{e_r}
    + \frac{\partial w}{\partial r}\hat{e_\phi} \bigg\}\\
  &+ \frac{\rho w}{r} \bigg\{
    \frac{\partial u}{\partial \phi}\hat{e_x} 
    + \left(\frac{\partial v}{\partial \phi}-r\right)\hat{e_r}
    + \left(\frac{\partial w}{\partial \phi}+v\right)\hat{e_\phi} \bigg\} \\
  &= \tau_{xx}\hat{e_x}
    + \bigg\{\tau_{xr}\hat{e_r}
    + \tau_{x\phi}\hat{e_\phi} \\
     \overrightarrow{\tau_2} 
    &= \tau_{rx}\hat{e_x}
    + \tau_{rr}\hat{e_r}
    + \tau_{r\phi}\hat{e_\phi} \\
     \overrightarrow{\tau_3} 
    &= \tau_{\phi x}\hat{e_x}
    + \tau_{\phi r}\hat{e_r}
    + \tau_{\phi\phi}\hat{e_\phi} \bigg\}
  \end{aligned}
\end{equation}

\subsection{Other Forms}
\begin{align}
      \pdv{\rho}{t}+\pdv{(\rho u_i)}{x_i} &= 0 \\
      \pdv{(\rho u_i)}{t}+\pdv{(\rho u_i u_j)}{x_j} &= -\pdv{p}{x_i}+\pdv{\tau_{ij}}{x_j}+\rho f_i \\
      \pdv{(\rho e)}{t}+(\rho e+p)\pdv{u_i}{x_i} &= \pdv{(\tau_{ij} u_j)}{x_i}+\rho f_i u_i+\pdv{(\dot{q}_i)}{x_i}+r \\
      \intertext{Classic notation}
      \vec{\nabla}\cdot(\rho\vec{u}) &= 0 \\
      \pdv{(\rho \vec{u})}{t}+\vec{\nabla}\cdot\rho\vec{u}\otimes\vec{u} &= -\vec{\nabla p}+\vec{\nabla}\cdot\Bar{\Bar{\tau}}+\rho\vec{f} \\
      \pdv{(\rho e)}{t}+\vec{\nabla}\cdot(\rho e+p)\vec{u} &= \vec{\nabla}\cdot(\Bar{\Bar{\tau}}\cdot\vec{u})+\rho\vec{f}\vec{u}+\vec{\nabla}\cdot\vec{\dot{q}}+r 
\end{align}

\subsection{Einstein summation convention}

\begin{equation}
\frac{\partial \rho}{\partial t} + \frac{\partial(\rho u_{i})}{\partial x_{i}} = 0
\end{equation}

\begin{equation}
\frac{\partial (\rho u_{i})}{\partial t} + \frac{\partial[\rho u_{i}u_{j}]}{\partial x_{j}} = -\frac{\partial p}{\partial x_{i}} + \frac{\partial \tau_{ij}}{\partial x_{j}} + \rho f_{i} \end{equation}
\begin{equation}
\frac{\partial (\rho e)}{\partial t} + (\rho e+p)\frac{\partial u_{i}}{\partial x_{i}} = \frac{\partial(\tau_{ij}u_{j})}{\partial x_{i}} + \rho f_{i}u_{i} + \frac{\partial(\dot{ q_{i}})}{\partial x_{i}} + r \end{equation}
The Einstein summation convention dictates that: When a sub-indice (here $i$ or $j$) is twice or more repeated in the same equation, one sums across the n-dimensions. 
This means, in the context of Navier-Stokes in 3 spacial dimensions, that one repeats the term 3 times, each time changing the indice for one representing the corresponding dimension (ie $1,2,3$ or $x,y,z$). Equation 1 is therefore a shorthand representation of: $\frac{\partial \rho}{\partial t}+\frac{\partial(\rho u_{1})}{\partial x_{1}}+\frac{\partial(\rho u_{2})}{\partial x_{2}}+ \frac{\partial(\rho u_{3})}{\partial x_{3}}=0$.
Equation $2$ is actually a superposition of 3 separable equations which could be written in a 3-line form: one line equation for each $i$ in each of which one sums the three terms for the $j$ sub-indice.
\subsection{Classic $\longrightarrow , \otimes , \nabla$ notation}
\begin{equation}
\frac{\partial \rho}{\partial t} + \overrightarrow{\nabla}\cdot(\rho\overrightarrow{u})=0 \end{equation}
\begin{equation}
\frac{\partial(\rho \overrightarrow{u})}{\partial t} + \overrightarrow{\nabla}\cdot[\rho\overline{\overline{u\otimes u}}] = -\overrightarrow{\nabla p} + \overrightarrow{\nabla}\cdot\overline{\overline{\tau}} + \rho\overrightarrow{f} \end{equation}
\begin{equation}
\frac{\partial(\rho e)}{\partial t} + \overrightarrow{\nabla}\cdot((\rho e + p)\overrightarrow{u}) = \overrightarrow{\nabla}\cdot(\overline{\overline{\tau}}\cdot\overrightarrow{u}) + \rho\overrightarrow{f}\overrightarrow{u} + \overrightarrow{\nabla}\cdot(\overrightarrow{\dot{q}})+r \end{equation}

Here $\otimes$ denotes the tensorial product, forming a tensor from the constituent vectors. A double bar denotes a tensor. The three equations ($4,5,6$) are equivalent to ($1,2,3$).
    

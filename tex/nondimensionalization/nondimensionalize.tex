\section{Non-Dimensionalization}

The Navier-Stokes equations presented in this report consist of a system of
nine equations in as many unknowns. These unknowns and their dimensional units
are summarized below:

\import{eqns/}{eq4_1}

where

\begin{equation}
\begin{aligned}
slug &= [lb sec^2/ft ] \\
N &= [kg m / sec^2 ]
\end{aligned}
\end{equation}

The conversion factors needed to from one dimensional system to the other are

\begin{equation}
\begin{aligned}
1 ft &= 0.3048 meter \\
1 lbf &= 4.4482216152605 Newton \\
1 ^o R &= \frac{9}{5} ^oK
\end{aligned}
\end{equation}

The dimensional properties above will non-dimensionalized by means of as yet
unspecified quantities. For the present, a * superscript will be used to denote
a dimensional quantity. The non-dimensional quantities will be defined as
follows:

\begin{center}
\begin{tabular}{ c c c }
$P = p^*/P^*_{ref}$ & $T = T^*/T^*_{ref}$ & $\rho = \rho^*/\rho^*_{ref}$ \\
$u = u^*/u^*_{ref}$ & $v = v^*/v^*_{ref}$ & $w = w^*/w^*_{ref}$ \\
$\mu = \mu^*/\mu^*_{ref}$ & $k = k^*/^*_{ref}$ & $H = h^*/h^*_{ref}$
\end{tabular}
\end{center}

In addition to these definitions, the length scales will be nondimensionalized
with some reference length.

\begin{equation}
l = l^*/l^*_{ref}
\end{equation}

The particular choice of the nondimensionalizing reference quantities is by no
means fixed and often depends on convention or tradition.However,in view of the
fact that most numerical studies closely couple with experimental efforts for
verification of results, it would appear appropriate to consider the quantities
available to an experimental researcher in establishing test conditions.

A wind tunnel, unless it has a variable nozzle, will be designed to operate at
a fixed Mach number. It will, therefore, operate at a fixed ratio of total to
static property ratios.In general, the driving total pressure is variable and,
in some cases, static temperature may be varied by using heaters, so these two
test conditions, together with the test Mach number, will be readily available
for any experimental study.

This information and the thermodynamic properties of the working fluid are
sufficient to completely describe the test conditions. For the present
numerical study, then, the following test conditions will be used to define the
reference quantities:

\begin{center}
\begin{tabular}{ c c }
Quantity & Units \\
\hline
$M_\infty$ & - \\ 
$P^*_o$ & $lb/ft^2$ \\
$T^*$ & $^oR$ \\
$R^*_{gas}$ & $ft^2/sec^2$ $^oR$ \\
$c_p^*$ & $ft/sec^2$ $^oR$ \\
$c_v^*$ & $ft^2/sec^2$ $^oR$
\end{tabular}
\end{center}

At this point, it would appear that the total quantities could be used for
reference parameters. However, by using the prefect gas relations, the static
quantities may br derived and used. Before making this decision, the governing
equations should be nondimensionalized so that the groupings of reference
quantities may be considered. Note that nondimensionalization of the length
scale gives the following:

\begin{equation}
\pdv{x} = l^*_{ref} \pdv{x^*}
\end{equation}

The angular derivatives will be left in their normal form. For the analysis
that follows, the pressure component of the normal shear stress terms will be
removed. That is:

\begin{equation}
\tau_{xx} = \sigma_{xx} - \pdv{P}{x} etc
\end{equation}

Finally, since the present work will deal only with the steady form of the
governing equations, the time derivatives will be eliminated . The
nondimensionalization follows:

\subsection{Continuity}

\begin{equation}
\frac{\rho^*_{ref}u^*_{ref}}{l^*_{ref}} \bigg\{
\pdv{x}\left( \rho u \right) 
+ \frac{1}{r}\pdv{r}\left( \rho v r \right)
+ \frac{1}{r}\pdv{\phi}\left(\rho w \right)
\bigg\} = 0
\end{equation}

\subsection{x Momentum}

\begin{equation}
  \frac{\rho^*_{ref}{u^*_{ref}}^2}{l^*_{ref}}\bigg\{
    \rho u \pdv{u}{x} + \rho v\pdv{u}{r} + \frac{\rho w}{r}\pdv{u}{\phi} 
  \bigg\}
\end{equation}

\subsection{r Momentum}

\begin{equation}
  \frac{\rho^*_{ref}{u^*_{ref}}^2}{l^*_{ref}}\bigg\{
    \rho u \pdv{v}{x} + \rho v\pdv{v}{r} + \frac{\rho w}{r}\left(\pdv{v}{\phi} - w\right) 
  \bigg\}
\end{equation}

\subsection{$\phi$ Momentum}

\begin{equation}
  \frac{\rho^*_{ref}{u^*_{ref}}^2}{l^*_{ref}}\bigg\{
    \rho u \pdv{w}{x} + \rho v\pdv{w}{r} + \frac{\rho w}{r}\left(\pdv{w}{\phi} + v\right) 
  \bigg\}
\end{equation}



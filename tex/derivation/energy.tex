\subsection{Energy Equation}

The energy equation follow from the principle conservation of energy.

\begin{quote} 
  The net rate of increase of the internal and kinetic energy, per
  unit mass, in $R$ is equal to the net flux of heat into $R$ through the
  surface $S$ due to heat conduction, plus the net rate of work done on $R$ due
  to body and surface forces, plus the net influx of energy into $R$ through
  $S$ due to the fluid motion.  
\end{quote}

It must be assumed that the classical   laws of thermodynamics, including the
first law, hold in the presence of shear stress and heat conduction in order
for the statement to be valid. if $e$ is the internal energy of the fluid per
unit mass, the net increase of the internal and kinetic energy is given by:

\begin{equation}
  \iiint_R\{\rho e + \rho \frac{V^2}{2}\} dR
\end{equation}

The net flux of heat into $R$ may be found by integrating the inward component
og the heat flux vector $\overrightarrow{Q}$ over the surface:

\begin{equation}
  \iint_S\overrightarrow{Q}\cdot\overrightarrow{dS}
\end{equation}

The work done on the fluid is a function of both body forces and shearing
forces, and is given by

\begin{equation}
  \iint_S\overrightarrow{V} \cdot\left(\rttensor{\tau}\cdot\overrightarrow{dS}\right) 
  + \iiint_R\left(rho \overrightarrow{f_g}\cdot\overrightarrow{V}\right)dR
\end{equation}

Finally, the net influx of energy due to the fluid motion is given by:

\begin{equation}
  \iint_S\left(\rho\overrightarrow{V}\cdot\overrightarrow{dS}\right)\left(e 
  + \frac{V+2}{2}\right)
\end{equation}

Combining these results, the energy equation becomes:

\begin{equation}
  \begin{split}
  \iiint_R\frac{\partial}{\partial{t}}\{\rho\left(e + \frac{v^2}{2}\right)dR = \\
    & - \iint_S\overrightarrow{Q}\cdot\overrightarrow{dS} \\
    & + \iiint_R\rho\overrightarrow{f_g}/cdot\overrightarrow{V} dR \\ 
    & + \iint_S\overrightarrow{V}\cdot\left(\rttensor{\tau}\cdot\overrightarrow{dS}\right) \\ 
    & - \iint_S\left(\rho\overrightarrow{V} \cdot\overrightarrow{dS}\right)\left(e + \frac{V^2}{2}\right) = 0
  \end{split}
\end{equation}

Which leads to this final result:

\begin{equation}
  \begin{split}
\frac{\partial}{\partial{t}}\rho\left
(e + \frac{V^2}{2}\right) \\
    & + \overrightarrow{\nabla}\cdot\overrightarrow{Q} \\
    & -\rho\overrightarrow{f_g}\cdot\overrightarrow{V} \\
    & - \overrightarrow{\nabla}\cdot\left(\rttensor{\tau}\cdot\overrightarrow{V}\right) \\
    & - \rho\overrightarrow{V}\cdot\overrightarrow{\nabla}\left(e + \frac{V^2}{2}\right) = 0
  \end{split}
\end{equation}

If the woring fluid is assumed to be Newtonian, the heat flux is a function of thermodynamic state only and may be given as:

\begin{equation}
  \overrightarrow{Q} =  \overrightarrow{\nabla}T
\end{equation}

where $k$ is the thermal conductivity of the fluid.



\subsection{Conservation of Momentum}

The conservation of momentum law is stated as follows:

\begin{quote}
  The time rate of momentum increase throughout {\bf R} is equal to the net force
  acting on {\bf R} plus the net flux of momentum into {\bf R} through the surface {\bf S}.
\end{quote}

The net force acting on {\bf R} will consist of any externally applied body force
plus the forces acting on {\bf R} that stem from the motion of the fluid itself.
These later forces are the internal stress forces acting on the fluid volume.
Since any body forces will be specified, we need to find a way to describe the
internal stress forces in the fluid at any point.  

Stress is related to the forces exerted on a fluid particle by adjacent
particles. If we cut our control volume with a plane, the force per unit area
acting on one side of the plane is caused by the fluid particles on the other
side of the plane. Since  the stresses measured at any point will be different
for each plane cut passing through that point, the minimum amount of
information necessary to completely describe the stress state at that point
must be determined.

Consider an arbitrary tetrahedron with three sides joining at a point in the
flow, and assume that the stress on these three sides is known. Then the
condition of state equilibrium can be used to find the stress on the fourth
side of this shape, regardless of the orientation of that fourth face with
respect to the other three. From this argument it may be seen that the stress
state at a point may be determined if the stresses on any three surfaces
passing through that point are known. For convenience, these three surfaces are
usually assumed to be orthogonal. Note that the three stresses are vector
quantities that can be expressed as a set of vector components. Furthermore,
the conditions of static equilibrium will result in a total of nine vector
components, but only six are independent.

The standard way to describe the stress state at a point in to set up a stress
matrix $\tau$ that looks like this:

\import{eqns/}{eq2_9}

In this notation $\tau_{xy}$ is the $y$ component of the stress acting on the
face perpendicular to the $x$ axis.

The three dependent stresses are:

\import{eqns/}{eq2_10}

Therefore, the stress matrix is seen to be symmetric.

In general, as a fluid volume moves, it will both deform and rotate. If we
consider two points along the path of this motion, we can use a {\it Taylor
Series} to relate the velocity at one point to the velocity at the other. In
Cartesian coordinates this becomes:

\import{eqns/}{eq2_11}

We can write similar equations for the remaining two velocity components. If we
only retail the linear terms, this reduced to:

\import{eqns/}{eq2_12}

This can be expressed in matrix form as:

\import{eqns/}{eq2_13}

This matrix is commonly referred to as the rate of deformation matrix.The
derivatives $\pdv{u}{x}$,
$\pdv{v}{y}$, and $\pdv{w}{z}$
indicate stretching motion called dilitation. The other derivatives indicate
distortion of the fluid due to shearing forces. By viewing the motion in this
manner it is apparent that the stresses in the fluid characterized by the
matrix $\rttensor{\tau}$ may be described by the above deformation matrix
together with the static pressure force. This analysis is known as {\it Stokes
Theorem} for stresses and is given below

As was previously noted, the stress matrix is symmetric. One method for forming
a symmetric matrix from $\Rttensor{D}$ is to combine it with its transpose. Thus
the stress tensor may be written as:

\import{eqns/}{eq2_14}

where $\mu$ is the bulk viscosity coefficient. It is common to define the
hydrostatic pressure $P$ as the mean value of the three normal stresses
$\tau_{xx}$, $\tau_{yy}$, and $\tau_{zz}$. In this case:

\import{eqns/}{eq2_15}

or

\import{eqns/}{eq2_16}

Therefore:

\import{eqns/}{eq2_17}

where $\lambda$ is the dilitational viscosity coefficient. This is related to
the bulk viscosity by {\it Stokes hypothesis}:

\import{eqns/}{eq2_18}

To find the contribution of the stress to the total forces acting on the
control volume, the shear force must be integrated over the surface of the
volume:

\import{eqns/}{eq2_19}

The total force acting on the fluid within the control volume is therefore given by:

\import{eqns/}{eq2_20}
where $\overrightarrow{f_g}$ is the externally applied force per unit mass.

The momentum flux through {\bf R} is given by:

\import{eqns/}{eq2_21}

The mathematical expression for the momentum equation may now be written as

\import{eqns/}{eq2_22}


Again, using the divergence theorem to convert surface integrals to volume
integrals, we get this:

\import{eqns/}{eq2_23}

As with the continuity equation, this leads to this final equation:

\import{eqns/}{eq2_24}



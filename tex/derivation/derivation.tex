\section{Derivation of Fluid Dynamics Equations of Motion}

The  system of equations which govern the flow of any fluid, the familiar
Navier-Stokes equations, may be derived from the concepts of conservation of
mass, momentum, and energy, together with the thermodynamic state equation for
the working fluid. Although this derivation is available from many sources, it
is included here for easy reference in other parts of this study.

Consider an arbitrary volume of fluid, {\bf R}, which is enclosed within a
surface, {\bf S}, with a unit outward normal {$\va{ds}$} as shown
below:

\importtikzfigure{control-volume}{Control Volume}

Let the vector {$\va{V}$} describe the velocity of fluid element
passing through the control volume surface at any time. Note that while the
shape of $R$ is arbitrary, it does not vary with time. Thus for any time
varying function:

\import{eqns/}{eq2_1}

\import{derivation/}{continuity}
\import{derivation}{momentum}
\import{derivation}{energy}
\import{derivation}{state}
\import{derivation}{summary}



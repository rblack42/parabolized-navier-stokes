\section{Derivation of Fluid Dynamics Equations of Motion}

The  system of equations which govern the flow of any fluid, the familiar
Navier-Stokes equations, may be derived from the concepts of conservation of
mass, momentum, and energy, together with the thermodynamic state equation for
the working fluid. Although this derivation is available from many sources, it
is included here for easy reference in other parts of this study.

Consider an arbitrary volume of fluid, {\bf R}, which is enclosed within a
surface, {\bf S}, with a unit outward normal {$\overrightarrow{ds}$} as shown
below:

\importtikzfigure{control-volume}{Control Volume}

Let the vector {$\overrightarrow{V}$} describe the velocity of fluid element
passing through the control volume surface at an y time. Note that while the
shape of $R$ is arbitrary, it does not vary with time. Thus for ant time
varying function:

\begin{equation}
\frac{\partial}{\partial t} \iiint_R (f) dr = \iiint_R \frac{\partial f}{\partial t}
\end{equation}

\subsection{Conservation of Mass}

Now, let $\overrightarrow{V}$ be the velocity of a fluid element passing
through the control volume $R$. We can state the principle of conservation of
mass as:

\begin{quote}
The time rate of change of mass increase within the control volume $R$, in
the absence of internal sources, is equal to the net flux of mass into $R$
through the surface $S$.
\end{quote}

The outward component of the mass flux at any point on $S$ is given by:

\begin{equation}
\rho \overrightarrow{V} \cdot \overrightarrow{ds}
\end{equation}

We can find the net mass flux into $R$ by integrating this expression over the
complete surface:

\begin{equation}
\iint_S{ - (\rho \overrightarrow{V}\cdot\overrightarrow{ds})}
\end{equation}

This surface integral can be converted into a volume integral by making use of
the *Divergence Theorem*:

\begin{equation}
\iint_S \overrightarrow{q}\cdot\overrightarrow{ds} = \iiint_R \left( \overrightarrow{\nabla} \cdot \overrightarrow{q}\right) dR
\end{equation}

Thus the total mass flux becomes:

\begin{equation}
-\iiint_R \left( \overrightarrow{\nabla}\cdot\rho\overrightarrow{V}\right) dR
\end{equation}

The conservation of mass law becomes:
    
\begin{equation}
\iiint_R \frac{\partial\rho}{\partial t} dR = -\iiint_R ( \overrightarrow{\nabla}\cdot\rho\overrightarrow{V}) dR
\end{equation}

or

\begin{equation}
\iiint_R \left[\frac{\partial\rho}{\partial t} + \overrightarrow{\nabla}\cdot\rho\overrightarrow{V}\right] dR = 0
\end{equation}

But, since the volume is arbitrary, the integrand must be zero, giving this final form:

\begin{equation}
\frac{\partial\rho}{\partial t} + \overrightarrow{\nabla}\cdot\rho\overrightarrow{V} = 0
\end{equation}

\subsection{Conservation of Momentum}

The conservation of momentum law is stated as follows:

\begin{quote}
The time rate of momentum increase throughout $R$ is equal to the net force
acting on $R$ plus the net flux of momentum into $R$ through the surface $S$.
\end{quote}

The net force acting on $R$ will consist of any externally applied body force
plus the forces acting on $R$ that stem from the motion of the fluid itself.
These later forces are the internal stress forces acting on the fluid volume.
Since any body forces will be specified, we need to find a way to describe the
internal stress forces in the fluid at any point.  

Stress is related to the forces exerted on a fluid particle by adjacent
particles. If we cut our control volume with a plane, the force per unit area
acting on one side of the plane is caused by the fluid particles on the other
side of the plane. Since  the stresses measured at any point will be different
for each plane cut passing through that point, the minimum amount of
information necessary to completely describe the stress state at that point
must be determined.

Consider an arbitrary tetrahedron with three sides joining at a point in the
flow, and assume that the stress on these three sides is known. Then the
condition of state equilibrium can be used to find the stress on the fourth
side of this shape, regardless of the orientation of that fourth face with
respect to the other three. From this argument it may be seen that the stress
state at a point may be determined if the stresses on any three surfaces
passing through that point are known. For convenience, these three surfaces are
usually assumed to be orthogonal. Note that the three stresses are vector
quantities that can be expressed as a set of vector components. Furthermore,
the conditions of static equilibrium will result in a total of nine vector
components, but only six are independent.

The standard way to describe the stress state at a point in to set up a stress
matrix $\tau$ that looks like this:

\begin{equation}
\tau = \begin{bmatrix}
\tau_{xx} & \tau_{xy} & \tau_{xz}\\
\tau_{yx} & \tau_{yy} & \tau_{yz} \\
\tau_{zx} & \tau_{zy} & \tau_{zz}
\end{bmatrix}
\end{equation}

In this notation $\tau_{xy}$ is the $y$ component of the stress acting on the
face perpendicular to the $x$ axis.

The three dependent stresses are:

\begin{align*}
\tau_{xy} &= \tau_{yx} \\
\tau_{xz} &= \tau_{zx} \\
\tau_{yz} &= \tau_{zy}
\end{align*}

Therefore, the stress matrix is seen to be symmetric.

In general, as a fluid volume moves, it will both deform and rotate. If we
consider two points along the path of this motion, we can use a *Taylor's
Series* to relate the velocity at one point to the velocity at the other. In
Cartesian coordinates this becomes:

\begin{equation}
\begin{aligned}
U_B = U_A & + \left(\frac{\partial U}{\partial x}\right)_A (x_B - x_A) \\
    & + \frac{1}{2}\left(\frac{\partial^2 U}{\partial x^2}\right)_A (x_b - x_A)^2 \\
    & + \left(\frac{\partial U}{\partial y}\right)_A (y_B - y_A) \\
    & + \frac{1}{2}\left(\frac{\partial^2 U}{\partial y^2}\right)_A (y_b - y_A)^2 \\
    & + \left(\frac{\partial U}{\partial z}\right)_A (z_B - z_A) \\
    & + \frac{1}{2}\left(\frac{\partial^2 U}{\partial z^2}\right)_A (z_b - z_A)^2 \\
    & + \left(\frac{\partial^2 U}{\partial x \partial y}\right)_A (x_B - x_A)(y_B - y_A) \\
    & + \left(\frac{\partial^2 U}{\partial x \partial z}\right)_A (x_B - x_A)(z_B - z_A) + \cdots
\end{aligned}
\end{equation}

We can write similar equations for the remaining two velocity components. If we only retail the linear terms, this reduced to:

\begin{equation}
U_B = U_A + \left(\frac{\partial U}{\partial x}\right)_A (x_B - x_A)  + \left(\frac{\partial U}{\partial y}\right)_A (y_B - y_A)  + \left(\frac{\partial U}{\partial z}\right)_A (z_B - z_A)
\end{equation}

This can be expressed in matrix form as:

\begin{equation}
  \Rttensor{D} = \begin{bmatrix}
  \frac{\partial{u}}{\partial{x}} & 
  \frac{\partial{u}}{\partial{y}} & 
  \frac{\partial{u}}{\partial{z}} \\

  \frac{\partial{v}}{\partial{x}} & 
  \frac{\partial{v}}{\partial{y}} & 
  \frac{\partial{v}}{\partial{z}} \\

  \frac{\partial{w}}{\partial{x}} & 
  \frac{\partial{w}}{\partial{y}} & 
  \frac{\partial{w}}{\partial{z}}
  \end{bmatrix} = \overrightarrow{\nabla}\overrightarrow{V}
\end{equation}

This matrix is commonly referred to as the rate of deformation matrix.The
derivatives $\frac{\partial{u}}{\partial{x}}$,
$\frac{\partial{v}}{\partial{y}}$, and $\frac{\partial{w}}{\partial{z}}$
indicate stretching motion called dilitation. The other derivatives indicate
distortion of the fluid due to shearing forces. By viewing the motion in this
manner it is apparent that the stresses in the fluid characterized by the
matrix $\rttensor{\tau}$ may be described by the above deformation matrix
together with the static pressure force. This analysis is known as {\it Stokes
Theorem} for stresses and is given below

As was previously noted, the stress matrix is symmetric. One method for forming
a symmetric matrix from $\Rttensor{D}$ is to combine it with its transpose. Thus
the stress tensor may be written as:

\begin{equation}
  \rttensor{\tau} = -P^* \overrightarrow{I} + \mu ( \Rttensor{D} + \Rttensor{D}^T )
\end{equation}

where $\mu$ is the bulk viscosity coefficient. It is common to define the
hydrostatic pressure $P$ as the mean value of the three normal stresses
$\tau_{xx}$, $\tau_{yy}$, and $\tau_{zz}$. In this case:

\begin{equation}
  P = \frac{1}{3} \{ -3P^* + 2\mu(\frac{\partial{u}}{\partial{x}} + \frac{\partial{v}}{\partial{y}} + \frac{\partial{u}}{\partial{x}} ) \}
\end{equation}

or

\begin{equation}
  P = -P^* + \frac{2}{3}\mu \nabla\cdot\overrightarrow{V}\overrightarrow{I} + \mu ( \Rttensor{D} + \Rttensor{D}^T )
\end{equation}

Therefore:

\begin{equation}
\rttensor{\tau} = -P\Rttensor{I} + \lambda\overrightarrow{\nabla}\cdot\overrightarrow{V}\Rttensor{I}  + \mu ( \Rttensor{D} + \Rttensor{D}^T )
\end{equation}

where $\lambda$ is the dilitational viscosity coefficient. This is relaed to
the bulk viscosity by Stokes hypothesis:

\begin{equation}
  \lambda =  - \frac{2}{3}\mu
\end{equation}

To find the contribution of the stress to the total forces acting on the
control volume, the shear force must be integrated over the surface of the
volume:

\begin{equation}
  d\overrightarrow{F_\tau} = \rttensor{\tau}\cdot \overrightarrow{dS}
\end{equation}

The total force acting on the fluid within the control volume is therefore given by:

\begin{equation}
  \overrightarrow{F} = \iiint_R\rho\overrightarrow{f_g} dR + \iint_S\rttensor{\tau}\cdot \overrightarrow{ds}
\end{equation}

where $\overrightarrow{f_g}$ is the externally applied force per unit mass.

The momentum flux through $R$ is given by:

\begin{equation}
  \iint_S \overrightarrow{V}\left(\rho\overrightarrow{V}\cdot\overrightarrow{dS}\right)
\end{equation}

The mathematical expression for the momentum equation may now be written as

\begin{equation}
  \iiint_R(\frac{\partial}{\partial{t}}(\rho\overrightarrow{V})dR = \iiint_R\rho\overrightarrow{f_g} dR + 
    \iint_S\rttensor{\tau}\cdot dS + \iint_S \overrightarrow{V}(\rho\overrightarrow{V}\cdot \overrightarrow{dS})
\end{equation}

Again, using the divergence theorem to convert surface integrals to volume integrals, we get this:

\begin{equation}
  \iiint_R\frac{\partial}{dt}(\rho\overrightarrow{V})dR = \iiint_R\{\rho \overrightarrow{f_g} + \overrightarrow{\nabla}\cdot\rttensor{\tau} - \overrightarrow{V}(\overrightarrow{\nabla}\cdot \rho\overrightarrow{V}) + \rho(\overrightarrow{V}\cdot\nabla)\overrightarrow{V}\}dR
\end{equation}

As with the continuity equation, this leads to this final equation:

\begin{equation}
\frac{\partial}{dt}\left(\rho\overrightarrow{V}\right) 
  - \rho \overrightarrow{f_g} 
  - \overrightarrow{\nabla}\cdot\rttensor{\tau} 
  + \overrightarrow{V}\left(\overrightarrow{\nabla}\cdot \rho\overrightarrow{V}\right) 
  + \rho\left(\overrightarrow{V}\cdot\nabla\right)\overrightarrow{V} = 0
\end{equation}

\subsection{Energy Equation}

The energy equation follow from the principle conservation of energy.

\begin{quote} 
  The net rate of increase of the internal and kinetic energy, per
  unit mass, in $R$ is equal to the net flux of heat into $R$ through the
  surface $S$ due to heat conduction, plus the net rate of work done on $R$ due
  to body and surface forces, plus the net influx of energy into $R$ through
  $S$ due to the fluid motion.  
\end{quote}

It must be assumed that the classical   laws of thermodynamics, including the
first law, hold in the presence of shear stress and heat conduction in order
for the statement to be valid. if $e$ is the internal energy of the fluid per
unit mass, the net increase of the internal and kinetic energy is given by:

\begin{equation}
  \iiint_R\{\rho e + \rho \frac{V^2}{2}\} dR
\end{equation}

The net flux of heat into $R$ may be found by integrating the inward component
og the heat flux vector $\overrightarrow{Q}$ over the surface:

\begin{equation}
  \iint_S\overrightarrow{Q}\cdot\overrightarrow{dS}
\end{equation}

The work done on the fluid is a function of both body forces and shearing
forces, and is given by

\begin{equation}
  \iint_S\overrightarrow{V} \cdot\left(\rttensor{\tau}\cdot\overrightarrow{dS}\right) 
  + \iiint_R\left(rho \overrightarrow{f_g}\cdot\overrightarrow{V}\right)dR
\end{equation}

Finally, the net influx of energy due to the fluid motion is given by:

\begin{equation}
  \iint_S\left(\rho\overrightarrow{V}\cdot\overrightarrow{dS}\right)\left(e 
  + \frac{V+2}{2}\right)
\end{equation}

Combining these results, the energy equation becomes:

\begin{equation}
  \begin{split}
  \iiint_R\frac{\partial}{\partial{t}}\{\rho\left(e + \frac{v^2}{2}\right)dR = \\
    & - \iint_S\overrightarrow{Q}\cdot\overrightarrow{dS} \\
    & + \iiint_R\rho\overrightarrow{f_g}/cdot\overrightarrow{V} dR \\ 
    & + \iint_S\overrightarrow{V}\cdot\left(\rttensor{\tau}\cdot\overrightarrow{dS}\right) \\ 
    & - \iint_S\left(\rho\overrightarrow{V} \cdot\overrightarrow{dS}\right)\left(e + \frac{V^2}{2}\right) = 0
  \end{split}
\end{equation}

Which leads to this final result:

\begin{equation}
  \begin{split}
\frac{\partial}{\partial{t}}\rho\left
(e + \frac{V^2}{2}\right) \\
    & + \overrightarrow{\nabla}\cdot\overrightarrow{Q} \\
    & -\rho\overrightarrow{f_g}\cdot\overrightarrow{V} \\
    & - \overrightarrow{\nabla}\cdot\left(\rttensor{\tau}\cdot\overrightarrow{V}\right) \\
    & - \rho\overrightarrow{V}\cdot\overrightarrow{\nabla}\left(e + \frac{V^2}{2}\right) = 0
  \end{split}
\end{equation}

If the woring fluid is assumed to be Newtonian, the heat flux is a function of thermodynamic state only and may be given as:

\begin{equation}
  \overrightarrow{Q} =  \overrightarrow{\nabla}T
\end{equation}

where $k$ is the thermal conductivity of the fluid.

\subsection{Remaining Equations}

The remaining equations neede to complete the set are the constituative relationships for the working fluid. In this investigation the fluid is a prefect gas and obeys the thermal state equation:

\begin{equation}
  P = \rho R T
\end{equation}

where $R$ is the ideal gas constant. If $c_p$ is the specific heat at constant pressure, and $c_v$ is the specific heat at constant volume, and if $\gamma = \frac{c_p}{c_v}$ are given, then the caloric state equation is

\begin{equation}
  e = c_v T
\end{equation}

The definition of the static enthalpy may be given by:

\begin{equation}
  h = c_p T = e + \frac{P}{\rho}
\end{equation}

Recalling the definition of the substantive derivative:

\begin{equation}
  \frac{Df}{Dt} = \frac{\partial f}{\partial t} + \overrightarrow{V}\cdot\nabla f
\end{equation}, the cmplete system of equations necessary to describe the flow of a fluid may be summarized:

\begin{enumerate}
  \item{$
    \frac{D\rho}{Dt} 
    + \rho \overrightarrow{\nabla}\cdot\overrightarrow{V} = 0
  $}
  \item{$
    \rho\frac{D\overrightarrow{V}}{Dt} = 
    \rho\overrightarrow{f_g} 
    + \overrightarrow{\nabla}\cdot\rttensor{\tau} 
  $}
  \item{$
    \rho\frac{D}{Dt}\left(e + \frac{V^2}{2}\right) =
    \rho \overrightarrow{f_g}\cdot\overrightarrow{V}
    - \overrightarrow{\nabla}\cdot\overrightarrow{Q}
    + \overrightarrow{\nabla}\cdot\left(\rttensor{\tau}\cdot\overrightarrow{V}\right)
  $}
  \item{$P = \rho R T $}
  \item{$e = c_v T $}
  \item{$h = c_p T $}
  \item{$ \gamma = \frac{c_p}{c_v} $}
  \item{$
    \rttensor{\tau} = 
    - P\overrightarrow{I} 
    + \lambda\left(\overrightarrow{\nabla}\cdot\overrightarrow{V}\right)\overrightarrow{I}
    + \mu\{\left(\overrightarrow{\nabla}\overrightarrow{V}\right) 
    + \left(\overrightarrow{\nabla}\overrightarrow{V}\right)^T\}
  $}
  \item{$ \overrightarrow{Q} = -k \overrightarrow{\nabla}T $}
  \item{$ \mu = \mu(T) $}
  \item{$\lambda = -\frac{2}{3}\mu $}
  \item{$k = k(T) $}
\end{enumerate}

\subsection{Einstein summation convention}

\begin{equation}
\frac{\partial \rho}{\partial t} + \frac{\partial(\rho u_{i})}{\partial x_{i}} = 0
\end{equation}

\begin{equation}
\frac{\partial (\rho u_{i})}{\partial t} + \frac{\partial[\rho u_{i}u_{j}]}{\partial x_{j}} = -\frac{\partial p}{\partial x_{i}} + \frac{\partial \tau_{ij}}{\partial x_{j}} + \rho f_{i} \end{equation}
\begin{equation}
\frac{\partial (\rho e)}{\partial t} + (\rho e+p)\frac{\partial u_{i}}{\partial x_{i}} = \frac{\partial(\tau_{ij}u_{j})}{\partial x_{i}} + \rho f_{i}u_{i} + \frac{\partial(\dot{ q_{i}})}{\partial x_{i}} + r \end{equation}
The Einstein summation convention dictates that: When a sub-indice (here $i$ or $j$) is twice or more repeated in the same equation, one sums across the n-dimensions. 
This means, in the context of Navier-Stokes in 3 spacial dimensions, that one repeats the term 3 times, each time changing the indice for one representing the corresponding dimension (ie $1,2,3$ or $x,y,z$). Equation 1 is therefore a shorthand representation of: $\frac{\partial \rho}{\partial t}+\frac{\partial(\rho u_{1})}{\partial x_{1}}+\frac{\partial(\rho u_{2})}{\partial x_{2}}+ \frac{\partial(\rho u_{3})}{\partial x_{3}}=0$.
Equation $2$ is actually a superposition of 3 separable equations which could be written in a 3-line form: one line equation for each $i$ in each of which one sums the three terms for the $j$ sub-indice.
\subsection{Classic $\longrightarrow , \otimes , \nabla$ notation}
\begin{equation}
\frac{\partial \rho}{\partial t} + \overrightarrow{\nabla}\cdot(\rho\overrightarrow{u})=0 \end{equation}
\begin{equation}
\frac{\partial(\rho \overrightarrow{u})}{\partial t} + \overrightarrow{\nabla}\cdot[\rho\overline{\overline{u\otimes u}}] = -\overrightarrow{\nabla p} + \overrightarrow{\nabla}\cdot\overline{\overline{\tau}} + \rho\overrightarrow{f} \end{equation}
\begin{equation}
\frac{\partial(\rho e)}{\partial t} + \overrightarrow{\nabla}\cdot((\rho e + p)\overrightarrow{u}) = \overrightarrow{\nabla}\cdot(\overline{\overline{\tau}}\cdot\overrightarrow{u}) + \rho\overrightarrow{f}\overrightarrow{u} + \overrightarrow{\nabla}\cdot(\overrightarrow{\dot{q}})+r \end{equation}

Here $\otimes$ denotes the tensorial product, forming a tensor from the constituent vectors. A double bar denotes a tensor. The three equations ($4,5,6$) are equivalent to ($1,2,3$).


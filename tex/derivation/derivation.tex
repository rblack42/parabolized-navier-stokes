\section{Derivation of Fluid Dynaics Equations of Motion}

The  system of equations which govern the flow of any fluid, the familiar
Navier-Stokes equations, may be derived from the concepts of conservation of
mass, momentum, and energy, together with the thermodynamic state equation for
the working fluid. Although this derivation is available from many sources, it
is included here for easy reference in other parts of this study.

Consider an arbitrary volume of fluid, {\bf R}, which is enclosed within a
surface, {\bf S}, with a unit outward normal {$\overrightarrow{ds}$} as shown
below:

\importtikzfigure{control-volume}{Control Volume}

Let the vector {$\overrightarrow{V}$} describe the velocity of fluid element
passing through the control volume surface at an y time. Note that while the
shape of $R$ is arbitrary, it does not vary with time. Thus for ant time
varying function:

\begin{equation}
\frac{\partial}{\partial t} \iiint_R (f) dr = \iiint_R \frac{\partial f}{\partial t}
\end{equation}

\subsection{Conservation of Mass}

Now, let $\overrightarrow{V}$ be the velocity of a fluid element passing
through the control volume $R$. We can state the principle of conservation of
mass as:

The time rate of change of mass increase within the control volume $R$, in
the absence of internal sources, is equal >to the net flux of mass into $R$
through the surface $S$.

The outward component of the mass flux at any point on $S$ is given by:

\begin{equation}
\rho \overrightarrow{V} \cdot \overrightarrow{ds}
\end{equation}

We can find the net mass flux into $R$ by integrating this expression over the
complete surface:

\begin{equation}
\iint_S{ - (\rho \overrightarrow{V}\cdot\overrightarrow{ds})}
\end{equation}

This surface integral can be converted into a volume integral by making use of
the *Divergence Theorem*:

\begin{equation}
\iint_S \overrightarrow{q}\cdot\overrightarrow{ds} = \iiint_R \left( \overrightarrow{\nabla} \cdot \overrightarrow{q}\right) dR
\end{equation}

Thus the total mass flux becomes:

\begin{equation}
-\iiint_R \left( \overrightarrow{\nabla}\cdot\rho\overrightarrow{V}\right) dR
\end{equation}

The conservation of mass law becomes:
    
\begin{equation}
\iiint_R \frac{\partial\rho}{\partial t} dR = -\iiint_R ( \overrightarrow{\nabla}\cdot\rho\overrightarrow{V}) dR
\end{equation}

or

\begin{equation}
\iiint_R \left[\frac{\partial\rho}{\partial t} + \overrightarrow{\nabla}\cdot\rho\overrightarrow{V}\right] dR = 0
\end{equation}

But, since the volume is arbitrary, the integrand must be zero, giving this final form:

\begin{equation}
\frac{\partial\rho}{\partial t} + \overrightarrow{\nabla}\cdot\rho\overrightarrow{V} = 0
\end{equation}

\subsection{Conservation of Momentum}

The conservation of momentum law is stated as follows:

The time rate of momentum increase throughout $R$ is equal to the net force
acting on $R$ plus the net flux of momentum into $R$ through the surface $S$.

The net force acting on $R$ will consist of any externally applied body force
plus the forces acting on $R$ that stem from the motion of the fluid itself.
These later forces are the internal stress forces acting on the fluid volume.
Since any body forces will be specified, we need to find a way to describe the
internal shear forces.  

Stress is related to the forces exerted on a fluid particle by adjacent
particles. If we cut our control volume with a plane, the force per unit area
acting on one side of the plane is caused by the fluid particles on the other
side of the plane. Since  the stresses measured at any point will be different
for each plane cut passing through that point, the minimum amount of
information necessary to completely describe the stress state at that point
must be determined.

Consider an arbitrary tetrahedron with three sides joining at a point in the
flow, and assume that the stress on these three sides is known. Then the
condition of state equilibrium can be used to find the stress on the fourth
side of this shape, regardless of the orientation of that fourth face with
respect to the other three. From this argument it may be seen that the stress
state at a point may be determined if the stresses on any three surfaces
passing through that point are known. For convenience, these three surfaces are
usually assumed to be orthogonal. Note that the three stresses are vector
quantities that can be expressed as a set of vector components. Furthermore,
the conditions of static equilibrium will result in a total of nine vector
components, but only six are independent.

The standard way to describe the stress state at a point in to set up a stress
matrix $\tau$ that looks like this:

\begin{equation}
\tau = \begin{bmatrix}
\tau_{xx} & \tau_{xy} & \tau_{xz}\\
\tau_{yx} & \tau_{yy} & \tau_{yz} \\
\tau_{zx} & \tau_{zy} & \tau_{zz}
\end{bmatrix}
\end{equation}

In this notation $\tau_{xy}$ is the $y$ component of the stress acting on the
face perpendicular to the $x$ axis.

The three dependent stresses are:

\begin{align*}
\tau_{xy} &= \tau_{yx} \\
\tau_{xz} &= \tau_{zx} \\
\tau_{yz} &= \tau_{zy}
\end{align*}

Therefore, the stress matrix is seen to be symmetric.

In general, as a fluid volume moves, it will both deform and rotate. If we consider two points along the path of this motion, we can use a *Taylor's Series* to relate the velocity at one point to the velocity at the other. In cartesian coordinates this becomes:

\begin{equation}
\begin{aligned}
U_B = U_A & + \left(\frac{\partial U}{\partial x}\right)_A (x_B - x_A) \\
    & + \frac{1}{2}\left(\frac{\partial^2 U}{\partial x^2}\right)_A (x_b - x_A)^2 \\
    & + \left(\frac{\partial U}{\partial y}\right)_A (y_B - y_A) \\
    & + \frac{1}{2}\left(\frac{\partial^2 U}{\partial y^2}\right)_A (y_b - y_A)^2 \\
    & + \left(\frac{\partial U}{\partial z}\right)_A (z_B - z_A) \\
    & + \frac{1}{2}\left(\frac{\partial^2 U}{\partial z^2}\right)_A (z_b - z_A)^2 \\
    & + \left(\frac{\partial^2 U}{\partial x \partial y}\right)_A (x_B - x_A)(y_B - y_A) \\
    & + \left(\frac{\partial^2 U}{\partial x \partial z}\right)_A (x_B - x_A)(z_B - z_A) + \cdots
\end{aligned}
\end{equation}

We can write similar equations for the remaining two velocity components. If we only retail the linear terms, this reduced to:

\begin{equation}
U_B = U_A + \left(\frac{\partial U}{\partial x}\right)_A (x_B - x_A)  + \left(\frac{\partial U}{\partial y}\right)_A (y_B - y_A)  + \left(\frac{\partial U}{\partial z}\right)_A (z_B - z_A)
\end{equation}

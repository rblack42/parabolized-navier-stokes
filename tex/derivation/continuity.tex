\subsection{Conservation of Mass}

Let $\va{V}$ be the velocity of a fluid element passing
through the control volume $R$. We can state the principle of conservation of
mass as:

\begin{quote}
  The time rate of change of mass increase within the control volume {\bf R}, in
  the absence of internal sources, is equal to the net flux of mass into {\bf R}
  through the surface {\bf S}.
\end{quote}

The outward component of the mass flux at any point on {\bf S} is given by:

\import{eqns/}{eq2_2}

Thus, the total mass flux through {\bf S} is given by the integral:

\import{eqns/}{eq2_3}

This surface integral can be converted into a volume integral by making use of
the {\it Divergence Theorem}:

\import{eqns/}{eq2_4}

Therefore, the mass flux becomes:

\import{eqns/}{eq2_5}

The conservation of mass law becomes:
    
\import{eqns/}{eq2_6}

or

\import{eqns/}{eq2_7}

But, since the {\bf R} is arbitrary, the above expression will only be valid if
the integrand is everywhere zero. That is:

\import{eqns/}{eq2_8}



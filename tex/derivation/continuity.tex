\subsection{Conservation of Mass}

Now, let $\va{V}$ be the velocity of a fluid element passing
through the control volume $R$. We can state the principle of conservation of
mass as:

\begin{quote}
The time rate of change of mass increase within the control volume $R$, in
the absence of internal sources, is equal to the net flux of mass into $R$
through the surface $S$.
\end{quote}

The outward component of the mass flux at any point on $S$ is given by:

\import{eqns/}{eq2}

We can find the net mass flux into $R$ by integrating this expression over the
complete surface:

\import{eqns/}{eq3}

This surface integral can be converted into a volume integral by making use of
the *Divergence Theorem*:

\import{eqns/}{eq4}

Thus the total mass flux becomes:

\import{eqns/}{eq5}

The conservation of mass law becomes:
    
\import{eqns/}{eq6}

or

\import{eqns/}{eq7}

But, since the volume is arbitrary, the integrand must be zero, giving this final form:

\import{eqns/}{eq8}


